A fundamental problem in algebraic complexity theory is determining the complexity of matrix multiplication. This problem concerns determining the smallest number of arithmetic operations required to multiply matrices, that is performing the map 
\[
\begin{aligned}
\MaMu(n_1,n_2,n_3) : \Mat_{n_1 \times n_2} \times \Mat_{n_2 \times n_3} &\to \Mat_{n_1 \times n_3} \\
(A , B) &\mapsto AB,
\end{aligned}
\]
as a function of $n_1,n_2,n_3$. Write $\MaMu(n)$ if $n=n_1=n_2=n_3$.

When $n = n_1=n_2=n_3$, the standard row-by-column multiplication algorithm shows that $\MaMu(n,n,n)$ can be performed using (at most) $O(n^3)$ arithmetic operations: specifically, each of the $n^2$ entries of $AB$ is computed performing $n$ multiplications and $n-1$ additions between scalars. In 1969, V. Strassen \cite{Str69} found an algorithm to evaluate $\MaMu(2,2,2)$ which only requires $7$ multiplications rather than $2^3 = 8$; we refer to \cite[Section 1.1.1]{Lan12} for the explicit expression. This started a long line of research both on upper and lower bounds on the complexity of $\MaMu(n,n,n)$. 

The asymptotic complexity is controlled by a universal constant, called the \emph{exponent of matrix multiplication}:
\[
\omega = \inf \bigl\{ \tau : \MaMu(n) \text{ can be evaluated using $n^\tau$ arithmetic operations} \bigr\}.
\]
The map $\MaMu(n_1,n_2,n_3)$ is bilinear, hence it defines a tensor 
\[
\MaMu(n_1,n_2,n_3) \in \Mat_{n_1 \times n_2}^* \otimes \Mat_{n_2 \times n_3}^* \otimes \Mat_{n_1 \times n_3}.
\]
A fundamental property of the matrix multiplication tensor is its \emph{self-reproducibility}, which encodes geometrically the fact that matrix multiplication can be performed in blocks.
\begin{lemma}
\label{matrixMultiplication-lemma-kroneckerproducts}
% Author: Fulvio Gesmundo
Let $n_1,n_2,n_3,m_1,m_2,m_3$ be nonnegative integers. Then 
 \[
\MaMu(n_1,n_2,n_3) \boxtimes \MaMu(m_1,m_2,m_3) = \MaMu(n_1m_1,n_2m_2,n_3m_3) 
\]
under suitable identifications $\Mat_{n_i \times n_{i+1}} \otimes \Mat_{m_i \times m_{i+1}} \simeq \Mat_{n_im_i \times n_{i+1}m_{i+1}}$.
\end{lemma}
The proof of \ref{matrixMultiplication-lemma-kroneckerproducts} will be evident from a more invariant point of view on the matrix multiplication tensor.

An important consequence of \ref{matrixMultiplication-lemma-kroneckerproducts}, and of general results on the computational complexity of multilinear ranks \cite[Ch.14]{BCS97}, is that the exponent of matrix multiplication is controlled by the tensor rank of the matrix multiplication tensor.
\begin{theorem}
\label{matrixMultiplication-thm-exponentasrank}
% Author: Fulvio Gesmundo
The exponent of matrix multiplication is characterized as 
\[
\omega = \inf \{ \tau : \rank(\MaMu(n))  \leq n^\tau \} .
\]
Equivalently, for every fixed $n$,
\[
\omega = \lim_{N \to \infty} \Bigl[ \frac{1}{N} \log_n \bigl(\rank  (\MaMu(n^N) ) \bigr) \Bigr].
\]
\end{theorem}


